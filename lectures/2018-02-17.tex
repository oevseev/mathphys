\dateline{17 февраля 2018 г.}

\section*{Введение. Организационные моменты}

\textbf{\color{blue}$\rightarrow$ Лектор --- Степанов Евгений Олегович.}

\begin{quote}
  Вроде как считается, что на матмехе чуть больше уровень математической подготовки, и в этом должен быть ваш плюс.
\end{quote}

Курс скорее должен называться <<Уравнения в частных производных>> --- текущее название по большей части обусловлено историческими причинами.

\subsection*{Устройство курса}

Обычный экзамен. Коллоквиума не будет; на 5 нужно сдать еще и задачу. Две (<<реально легкие>>) теоретические к/р, гарантирующие автомат (3 или 4) при нужном количестве баллов. Чтобы сдать на 5, нужно сдавать экзамен.

\subsection*{Необходимые знания}

\begin{itemize}
  \item Анализ вещественной переменной;
  \item Основные определения топологии;
  \item Мера Лебега и интеграл Лебега. Базовые определения:
    \begin{itemize}
      \item $\Omega \in \mathbb{R}^n$ --- область ($\equiv$ открытое множество),
      \item $L^1(\Omega)$ --- класс функций, измеримых по Лебегу,
      \item $L^p(\Omega) \coloneqq \{ u \colon \int_\Omega \abs{u(x)}^p \dd x < +\infty \}$,
      \item $L^\infty(\Omega) \coloneqq \{ u \colon \exists C > 0 \colon \abs{u(x)} \le c\ \text{почти всюду} \}$;
    \end{itemize}
  \item <<Неплохо бы уметь интегрировать по частям>>;
  \item Умение работать с частными производными.
\end{itemize}

\clearpage

\section{Уравнения в частных производных}

\subsection{Модель распространения загрязнения}

\subsubsection{Модель}

Есть пятно воды в одномерной реке (изначально представляющее собой отрезок в $\mathbb{R}$), текущей со скоростью $v$.

\begin{figure}[ht]
  \centering
  \placeholder{см. тетрадь}
  \caption{Иллюстрация модели}
\end{figure}

На отрезке $[x, x + \Delta x]$ общая масса вещества будет равна $\int_x^{x + \Delta x} c(t, \xi) \dd \xi$, где $c(t, x)$ --- концентрация вещества в точке $x$ в момент времени $t$.

Обозначим поток через границу направо через $q(t, x)$. Концентрация $c(t, x)$ подчиняется закону сохранения массы. Тогда справедливо:
%
\begin{equation}
  \dv{t} \int_x^{x + \Delta x} c(t, \xi) \dd \xi = -q(t, x + \Delta x) + q(t, x)
\end{equation}

Отсюда:
%
\begin{equation}
  \int_{x}^{x + \Delta x} \pdv{c}{t} (t, \xi) \dd \xi = \frac{q(t, x) - q(t, x + \Delta x)}{\Delta x}
\end{equation}

Устремляя $\Delta x \to 0$, получаем следующий закон:
%
\begin{equation}
  \pdv{c}{t} (t, x) = - \pdv{q}{x} (t, x)
\end{equation}

В общем виде этот процесс называется \emph{диффузией} и выглядит следующим образом:
%
\begin{equation}
  q(t, x) = k \pdv{c}{x} (t, x)
\end{equation}
%
где $k = -D$, $D > 0$ --- \emph{коэффициент диффузии (коэффициент Дарси)}.

Дифференцируя еще раз по $x$, получаем \emph{уравнение диффузии} (<<в скобочках --- уравнение теплопроводности>>):
%
\begin{equation}
  \pdv{c}{t} (t, x) = D \pdv{c}{x}{x} (t, x)
\end{equation}

Возможен следующий случай: диффузии нет (чистый снос материала жидкостью). Тогда поток равен:
%
\begin{equation}
  q(t, x) = vc(t, x)
\end{equation}

Дифференцируя по $t$, получаем \emph{транспортное уравнение/закон сохранения массы при конвекции/уравнение переноса}:
%
\begin{equation}
  \pdv{c}{t} (t, x) = tx \pdv{c}{x} (t, x) \label{transport}
\end{equation}

Это линейное уравнение 2-го порядка.

Возможен и третий случай (диффузия + конвекция). Тогда справедливо следующее:
%
\begin{equation}
  q(t, x) = -D \pdv{c}{x} (t, x) + v c(t, x)
\end{equation}

Можно вывести \emph{уравнение диффузии с конвективным членом}:
%
\begin{equation}
  \pdv{c}{t}(t, x) = D \pdv{c}{x}{x} (t, x) - v \pdv{c}{x} (t, x)
\end{equation}

\subsubsection{Решение транспортного уравнения}

Требуется найти решение транспортного уравнения \eqref{transport}:
%
\begin{equation}
  \pdv{c}{t} + v \pdv{c}{x} = 0 \label{transport2}
\end{equation}
%
с начальными условиями $c(0, x) = c_0(x)$ (классическое решение --- \placeholder{уточнить}).

Каждая частица <<этой массы>> эволюционирует следующим образом:
%
\begin{equation}
  \begin{cases}
    \dv{x}{t} = v \\
    \eval{x}_{t=0} = x_0
  \end{cases}
\end{equation}

Есть догадка $c(t, x(t))$ --- константа. Докажем это аналитически. Продифференцируем:
%
\begin{equation}
  \dv{t} c(t, x(t)) = \pdv{c}{t} (t, x(t)) + \pdv{c}{x} (t, x(t)) \cdot \dv{x}{t}
\end{equation}

Согласно \eqref{transport2}, правая часть равна\footnote{А еще $\dv{x}{t} = v$, если это вдруг не очевидно} $0$.

Можно проиллюстрировать решение следующим образом:
%
\begin{figure}[ht]
  \centering
  \placeholder{см. тетрадь}
  \caption{Иллюстрация решения}
\end{figure}

Классическое решение выглядит следующим образом:
%
\begin{equation}
  c(t, x) = c_0(x - vt)
\end{equation}

\begin{quote}
  Если кто-то во все-это не верит/не понял/не запомнил --- наплевать, берем формулу и подставляем.
\end{quote}

\begin{thm}
  Если $c_0 \in C^1(\mathbb{R})$, то $c(t, x) = c_0(x - vt)$ --- это единственное решение\footnote{<<Под решением здесь все еще подразумевается классическое>>} задачи Коши \eqref{transport2}.
\end{thm}

Такое решение называется решением бегущей волны (travelling wave):
%
\begin{figure}[ht]
  \centering
  \placeholder{см. тетрадь}
  \caption{Travelling wave}
\end{figure}

\subsubsection{Решение задачи с начальными условиями}

Рассмотрим теперь другую задачу:
%
\begin{equation}
  \begin{cases}
    \pdv{c}{t} + v \pdv{c}{x} = f(t, x) \\
    c(0, x)) = c_0(x)
  \end{cases} \label{cauchy2}
\end{equation}

Рассмотрим производную $c(t, x(t))$:
%
\begin{equation}
  \dv{t} c(t, x(t)) = \pdv{c}{t} + \pdv{c}{x} \cdot \dv{x}{t} = \pdv{c}{t} + v \pdv{c}{x} = f(t, x)
\end{equation}

В частности, $\dv{t} c(t, x_0 + vt) = f(t, x_0 + vt)$, откуда:
%
\begin{equation}
  c(t, x_0 + vt) = c(0, x_0) + \int_0^t f(s, x_0 + vs) \dd s
\end{equation}

Так как $x_0 + vt = x$, имеем $x_0 = x - vt$. Подставляем:
%
\begin{equation}
  \begin{aligned}
  c(t, x) &= c(0, x - vt) + \int_0^t f(s, x - vt + vs) \dd s\\
  &= c_0(x - vt) + \int_0^t f(s, x - v(t - s)) \dd s
  \end{aligned} \label{sol-cauchy2}
\end{equation}

Сформулируем получившийся результат в виде теоремы:
%
\begin{thm}
  Если $c_0 \in C^1(\mathbb{R})$, $f \in C(\mathbb{R}^+, \mathbb{R})$ и $\pdv{f}{x} \in C(\mathbb{R}^+, \mathbb{R})$, то \eqref{sol-cauchy2} --- единственное (классическое) решение \eqref{cauchy2}.
\end{thm}

\subsection{Основная лемма вариационного исчисления}

\subsubsection{Формулировки}

\begin{lem}\emph{(лемма Дюбуа-Реймона, слабая формулировка)}
  $\Omega \in \mathbb{R}^n$ --- область, $f \in C(\Omega)$ и выполняется \begin{equation}
    \int_\Omega f(x) g(x) \dd x = 0 \label{dubois-condition}
  \end{equation}
  для любых $g \in C_0^\infty(\Omega)$.\footnote{Определим $\supp g \coloneqq \overline{\{x \colon g(x) \neq 0 \}}$ --- носитель функции. Тогда $C_0^k$ --- семейство функций с компактным носителем, у которых первые $k$ производных непрерывны.} Тогда $f \equiv 0$.
\end{lem}

\begin{lem}\emph{(лемма Дюбуа-Реймона, сильная формулировка)}
  $\Omega \in \mathbb{R}^n$ --- область, {\color{red} $f \in L^1_{\mathrm{loc}}(\Omega)$} и выполняется \eqref{dubois-condition} для любых $g \in C_0^\infty(\Omega)$. Тогда $f \equiv 0$ {\color{red} почти всюду в $\Omega$}.
\end{lem}

\subsubsection{Доказательство леммы в слабой формулировке}

\begin{proof}
  От противного. Пусть $f \in C(\Omega)$ удовлетворяет \eqref{dubois-condition} и $\exists x_0 \in \Omega \colon f(x_0) \neq 0$ (НУО $f(x_0) > 0$). В силу непрерывности $f$ существует $r > 0$ такое, что $f(x) > 0\ \forall x \in B_r(x_0) \subset \Omega$.
  
  Возьмем функцию $g(x) \in C^\infty(\Omega)$, которая бы всюду на $B_r(x_0)$ больше $0$, а всюду вне $B_r(x_0)$ --- равна $0$. Ясно, что $\supp g = \overline{B_r(x_0)} \Subset \Omega$,\footnote{$A \Subset B$ означает, что $A$ --- компактное подмножество $B$} тем самым $g \in C_0^\infty(\Omega)$.
  
  Мы знаем, что по условию:
  \begin{equation}
    \int_\Omega f(x) g(x) \dd x = 0
  \end{equation}
  
  С другой стороны:
  \begin{equation}
    \begin{aligned}
      \int_\Omega f(x) g(x) \dd x &= \int_{B_r(x_0)} f(x) g(x) \dd x + \int_{\Omega \setminus B_r(x_0)} f(x) g(x) \dd x \\
      &= \int_{B_r(x_0)} f(x) g(x) \dd x \\
      &> 0
    \end{aligned} \label{estimation}
  \end{equation}
  Последнее справедливо в силу того, что обе функции положительны на области ненулевой меры.
  
  Осталось доказать, что нужная функция $g$ существует. Определим $\psi \colon \mathbb{R} \to \mathbb{R}$:
  %
  \begin{equation}
    \psi(x) \coloneqq \begin{cases}
      e^{1/\qty(x^2 - 1)} & x \in (-1, 1) \\
      0 & \text{иначе}
    \end{cases}
  \end{equation}
  
  График такой функции выглядит следующим образом:
  \begin{figure}[ht]
    \centering
    \placeholder{см. тетрадь}
    \caption{График функции $\psi(x)$}
  \end{figure}

  Положим $g(x) \coloneqq \psi(\abs{x - x_0} / r)$. Нетрудно показать, что она действительно удовлетворяет необходимым условиям.
  
  Таким образом, получаем, что \eqref{estimation} противоречит \eqref{dubois-condition}. Теорема доказана.
\end{proof}

\subsection{Колебания струны и волновое уравнение}

Хотелось бы описать колебания струны уравнением.

\subsubsection{Модель и формулировка задачи}

Пусть у нас есть струна, выведенная из состояния равновесия. Струна определена на отрезке и описывается координатой $x$ и смещением отностительно состояния равновесия $u = u(t, x)$:
%
\begin{figure}[ht]
  \centering
  \placeholder{см. тетрадь}
  \caption{Иллюстрация модели}
\end{figure}

Есть принцип, сформулированный в свое время еще Лагранжем (принцип минимального действия), как описать динамику механической системы:
\begin{equation}
  A = \int_0^\tau (T - U) \dd t \longrightarrow \min
\end{equation}
где $A$ --- действие, $T$ --- кинетическая энергия, $U$ --- потенциальная энергия.

Кинетическая энергия одного маленького отрезка струны равна $\frac{u_t^2}{2} \rho_0 \Delta x$, откуда:
\begin{equation}
  T = \int_0^L \rho_0 \frac{u_t^2}{2} \dd x.
\end{equation}

Работа одного маленького отрезка равна:
%
\begin{equation}
  \begin{aligned}
  \int_x^{x + \Delta x} \sqrt{1 + u_x^2}\ \dd x - \Delta x &= \int_x^{x + \Delta x} \sqrt{1 + y_x^2 - 1}\ \dd x \\
  &\approx \int_x^{x + \Delta x} \left(1 + \frac{u_x^2}{2} - 1\right) \dd x \\
  &= \frac{1}{2} u_x^2 \Delta x
  \end{aligned}
\end{equation}
где приближение обусловлено тем, что квадрат смещения струны пренебрежимо мал.

Работа всей струны равна:
%
\begin{equation}
  \begin{aligned}
    A &= \int_0^\tau \dd t \left(\int_0^L \frac{\rho_0}{2} u_t^2 \dd x - \int_0^L \frac{\tau_0}{2} u_x^2 \dd x \right) \\
    &= \frac{1}{2} \int_0^\tau \dd t \int_0^L \dd x \left(\rho_0 u_t^2 - \tau_0 u_x^2 \right)
  \end{aligned}
\end{equation}