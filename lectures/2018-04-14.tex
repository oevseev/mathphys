\dateline{14 апреля 2018 г.}

Рассмотрим функционал:
\begin{equation}
  F(u) \coloneqq \frac{1}{2} \int_\Omega \abs{\nabla u}^2 \dd x - \int_\Omega fu \dd x
\end{equation}
Здесь $u \in H_0^1(\Omega)$, где $\Omega \subset \mathbb{R}^n$ открытое и \underline{ограниченное}\footnote{нужно для того, чтобы применить неравенство Фридрихса}, $f \in L^2(\Omega)$.

\begin{thm}
  Существует такое $u_0 \in H_0^1(\Omega)$, что $F(u_0) = \inf_{H_0^1(\Omega)} F$.
\end{thm}

Ранее был рассмотрен другой функционал на пространстве гладких функций, но доказать существование минимума не удавалось. В случае соболевских пространств это делается легко:

\begin{proof}
  Рассмотрим минимизирующую последовательность $u_k \in H_0^1(\Omega)$ и $F(u_k)$ убывает к $\inf_{H_0^1(\Omega)} F$.
  
  Можем расписать $F(u_k)$:
  \begin{equation}
    F(u_k) = \frac{1}{2} \norm{\nabla u_k}^2_2 - \int_\Omega f u_k \dd x \label{20180414-kek}
  \end{equation}
  
  Интеграл поделим и умножим на $\eps$, а затем оценим сверху:
  \begin{equation}
    \int_\Omega f u_k \dd x = \frac{1}{\eps} \int_\Omega \eps u_k \dd x \le \int_\Omega \frac{1}{2} \qty(\qty(\frac{f}{e})^2 + \qty(\eps u_k)^2) \dd x
  \end{equation}
  
  Это позволяет нам в свою очередь оценить \eqref{20180414-kek}:
  \begin{equation}
    F(u_k) \ge \frac{1}{2} \norm{\nabla u_k}^2_2 - \frac{1}{2} \eps^2 \int_\Omega f^2 - \eps^2 \norm{u_k}_2^2 \label{20180414-kek2}
  \end{equation}
  
  В то же время, по неравенству Фридрихса:
  \begin{equation}
    \norm{u_k}^2_2 \le C(\Omega) \norm{\nabla u_k}^2_2
  \end{equation}
  
  Что позволяет на продолжить оценку \eqref{20180414-kek2}:
  \begin{equation}
    \text{\placeholder{глянуть выкладку}}
  \end{equation}
  
  Если мы подставим $0$ в наш функционал, получим следующее неравенство:
  \begin{equation}
    \frac{1}{2}(1 - \eps^2 C(\Omega)) \norm{\nabla u_k}_2^2 \le \frac{1}{2 \eps^2} \norm{f}_2^2
  \end{equation}
  и это для любого $\eps$. Следовательно, если мы подставим хорошее \placeholder{(какое?)} $\eps$, можем получить следующую оценку, сохранив неравенство:
  \begin{equation}
    \norm{\nabla u_k}^2_2 \le C(\norm{F}_2, \Omega)
  \end{equation}
  
  В свою очередь,
  \begin{equation}
    \norm{u_k}^2_2 \le C(\Omega) \norm{\nabla u_k}_2^2 \le C
  \end{equation}
  
  Что мы получили? Если $u_k$ --- минимизирующая, то вся она находится в каком-то шаре пространства $H_0^1$.
  
  Перейдем к следующему шагу. Используем факт, что $H_0^1$ --- сепарабельное гильбертово пространство, а конкретно --- свойство, что из любой последовательности в шаре этого пространства можно выбрать слабо сходящуюся подпоследовательность.
  
  Пусть $u_{k_m} \to u \in H_0^1(\Omega)$ слабо. Рассмотрим вложение $i \colon H_0^1(\Omega) \to L^2(\Omega)$. Проверим ограниченность оператора вложения. \placeholder{Проверили.} Из этого следует, что $u_{k_m} \to u$ слабо в смысле $L^2(\Omega)$.
  
  Мало того, давайте рассмотрим дополнительно следующие операторы обобщенного дифференцирования (их $n$ штук --- по размерности пространства):
  \begin{equation}
    \pdv{x_i} \coloneqq u \in H_0^1(\Omega) \mapsto \pdv{u}{x_i} \in L^2(\Omega)
  \end{equation}
  Они тоже ограничены, следовательно последовательность $\pdv{u_{k_m}}{x_i} \to \pdv{u}{x_i}$ в смысле $L^2(\Omega)$.
  
  Теперь устремим $m \to \infty$. Рассмотрим значения функционала на функциях $u_{k_m}$:
  \begin{equation}
    F(u_{k_m}) = \frac{1}{2} \int_\Omega \abs{\nabla u_{k_m}}^2 - \int_{\Omega} f u_{k_m}
  \end{equation}
  
  Посмотрим на второй интеграл. $u_{k_m} \to u$ в смысле $L^2(\Omega)$, следовательно интеграл, будучи линейным непрерывным функционалом на $L^2$, ведет себя следующим образом:
  \begin{equation}
    \int_\Omega f u_{k_m} \dd x \to \int_{\Omega} fu \dd x
  \end{equation}
  
  Чтобы убить первый интеграл, оценим $(\nabla u_k - \nabla u)^2$:
  \begin{equation}
    0 \le (\nabla u_k - \nabla u)^2 = \abs{\nabla u_k}^2 + \abs{\nabla u}^2 - 2 \nabla u_k \cdot \nabla u
  \end{equation}
  
  То есть:
  \begin{equation}
    \abs{\nabla u_k}^2 - \abs{\nabla u}^2 \ge 2 \nabla u \cdot (\nabla u_k - \nabla u)
  \end{equation}
  
  Поделим обе части неравенства на $2$ и проинтегрируем по $\Omega$, а также распишем скалярное произведение:
  \begin{equation}
    \frac{1}{2} \int_\Omega \abs{\nabla u_k}^2 - \frac{1}{2} \int_\Omega \abs{\nabla u}^2 \ge \int_\Omega \nabla u \cdot (\nabla u_{k_m} - \nabla u)
  \end{equation}
  
  Правую часть можем приравнять:
  \begin{equation}
    \text{\placeholder{выкладка}}
  \end{equation}
  
  В силу доказанного ранее можем перейти к пределу. Получим:
  \begin{equation}
    \lim \inf \frac{1}{2} \int_\Omega \abs{\nabla u_{k_m}}^2 - \int_{\Omega} \abs{\nabla u}^2 \ge 0
  \end{equation}
  
  Можем переформулировать это следующим образом.
  \begin{equation}
  \text{\placeholder{выкладка}}
  \end{equation}
  
  иначе говоря, мы докзали, что наш функционал полунепрерывен снизу.
  
  Завершим наше доказательство. Заметим, что $\lim_m \inf F(u_{k_m}) \to \inf_{H_0^1}(\Omega) F \ge F(u)$. Но $F(u)$ само из $H_0^1$, следовательно, получаем равенство.
\end{proof}

Какая связь между полученным результатом и уравнением Пуассона? Рассмотрим $F(u) = \inf_{H_0^1(\Omega)} \inf F$, где $u \in H_0^1(\Omega)$, и посмотрим, какому уравнению наше $u$ удовлетворяет.

Возьмем произвольную вариацию $\phi \in H_0^1(\Omega)$. Гарантированно имеем $F(u) \le F(u + \eps \phi)$.

\bigskip
\hrule
\smallskip
\placeholder{Пропущена часть лекции}
\hrule
\smallskip
\bigskip

Второе доказательство --- ни что иное, как приложение леммы к случаю конкретных пространств, а сама лемма является частным случаем теоремы Рисса. Доказательство с функционалом же конструктивно и позволяет нам понять, как выглядит решение.

На основе первого доказательства основан численный метод --- метод Ритца.