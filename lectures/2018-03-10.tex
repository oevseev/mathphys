\dateline{10 марта 2018 г.}

\setcounter{subsection}{4}

\subsection{Принцип максимума для уравнения теплопроводности}

\placeholder{Пропущена первая половина лекции}

\begin{proof}
  Зафиксируем $u$, возьмем $\eps > 0$. Положим
  \begin{equation}
    v_\eps(t, x) \coloneqq u(t, x) = e \abs{x}^2 \ge u(t, x)
  \end{equation}
  
  Применим $L$ к $v_\eps$:
  \begin{equation}
    Lv_\eps = \pdv{v_\eps}{t} - a^2 \Delta v_\eps = (\pdv{u}{t} - a^2 \Delta u) - \eps a^2 \Delta \underbrace{\abs{x}^2}_{2n} = -2\eps a^2 n < 0
  \end{equation}
  
  Докажем, что $v_\eps$ не достигает своего максимума на <<верхней крышке>> или внутри стакана:
  
  \begin{enumerate}
    \item Пусть максимум достигается на $(x_0, t_0)$, где $x_0 \in \Omega$, $t_0 < T$. Тогда:
    \begin{align}
      \pdv{v_\eps}{t}(t_0, x_0) &= 0 \\
      \pdv{v_\eps}{x_i}{x_i}(t_0, x_0) &\le 0
    \end{align}
    
    Отсюда:
    \begin{equation}
      (Lv_\eps)(t_0, x_0) = \pdv{v_\eps}{t} - a^2 \sum_i \pdv{v_\eps}{x_i^2} \eval_{t_0, x_0} \ge 0
    \end{equation}
    
    \item Пусть максимум достигается на... \placeholder{чем?} Считаем то же самое (\placeholder{что?})
  \end{enumerate}

  Заключаем, что не на параболической границе функция достигать своего максимума не можем. В итоге имеем\footnote{$S_T$ --- параболическая граница}:
  \begin{equation}
    u(t, x) \le v_\eps(t, x) \le \max_{\overline{Q_T}} v_\eps = \max_{S_T} v_\eps = \max_{S_T} (u + \eps\abs{x}^2) \le \max_{S_T} u + \eps \max_{S_T} \abs{x}^2
  \end{equation}
  
  Подводя итог, заключаем, что:
  \begin{equation}
    u(t, x) \le \max_{S_T} u + C\eps
  \end{equation}
  
  Теорема доказана.
\end{proof}

\subsubsection{Принцип максимума уравнения теплопроводности во всем пространстве}

Пусть $Lu = \pdv{u}{t} - a^2 \Delta u = 0$, $t \in \mathbb{R}^+$, $x \in \mathbb{R}^n$. ($u$ --- классическое решение, а, следовательно, непрерывно вплоть до $t = 0$).

{\color{red} Пусть дополнительно $\abs{u(t, x)} \le C_T$ для всех $(t, x) \in [0, T] \times \mathbb{R}^n$} --- <<стакан стал бесконечен и боковых стенок у него нет>>; параболическая граница --- $t = 0$ (вырожденный случай).

Введем следующие обозначения:
\begin{align}
  \begin{aligned}
    M_+ &\coloneqq \sup_{x \in \mathbb{R}^n, t \in [0, T]} u(t, x) \\
    M_- &\coloneqq \inf_{x \in \mathbb{R}^n, t \in [0, T]} u(t, x) 
  \end{aligned} \\
  \begin{aligned}
    N_+ &\coloneqq \sup_{x \in \mathbb{R}^n} u(0, x) \\
    N_- &\coloneqq \inf_{x \in \mathbb{R}^n} u(0, x)
  \end{aligned}
\end{align}

\begin{thm}
  $M_+ = N_+$.
\end{thm}

\begin{cor}
  $M_- = N_-$.
\end{cor}

\begin{proof}
  Применим теорему к $-u$.
\end{proof}

Поставим задачу Коши:
\begin{equation}
  \left\{\begin{aligned}
  u_t - a^2 \Delta u = f \\
  u(0, x) = u_0(x)
  \end{aligned}\right. \label{cauchy_max}
\end{equation}

Имеет место быть следующее следствие:

\begin{cor}
  Система \eqref{cauchy_max} не может иметь двух разных классических решений, удовлетворяющих ограничениям из начала параграфа.
\end{cor}

\begin{proof}
  Пусть $u_1, u_2$ --- два решения. Рассмотрим $u \coloneqq u_1 - u_2$. Тогда система \eqref{cauchy_max} превращается в:
  \begin{equation}
    \left\{\begin{aligned}
      Lu = Lu_1 - Lu_2 = f - f = 0 \\
      u(0, x) = u_1(0, x) - u_2(0, x) = 0
    \end{aligned}\right.
  \end{equation}
  
  Отсюда $u$ --- класс решений; при этом $u$ также лежит в необходимом классе функций. Следовательно, $\sup u = \inf u = \sup_{S_T} u = 0$, откуда $u$ --- тождественный ноль и $u_1 = u_2$.
\end{proof}

\begin{rem}
  Пример не единственности (неклассического) решения --- пример Тихонова.
\end{rem}

\begin{proof}(теоремы)
  Положим \placeholder{\sout{огромнейший болт на все это}}:
  \begin{equation}
    v_\eps(t, x) \coloneqq u(t, x) - \eps (2n a^2 t + \abs{x}^2)
  \end{equation}
  
  Применим $L$:
  \begin{equation}
    Lv_\eps = Lu - \eps L (2n a^2 t + \abs{x}^2) = -\eps (2na^2 - a^2 2n) = 0
  \end{equation}
  
  Рассмотрим цилиндр $\overline{Q_{R, T}} \coloneqq [0, T] \times \overline{B_R(0)}$.
  
  Оценим $v_\eps(0, x)$ сверху:
  \begin{equation}
    v_\eps(0, x) = u(0, x) - \eps \abs{x}^2 \le N_+
  \end{equation}
  
  Посмотрим на поведение $v_\eps$ на границе шара:
  \begin{equation}
    v_\eps(t, x) \eval_{x \in \fr B_R(0)} = u(t, x) \eval_{x \in \fr B_R(0)} = \eps R^2 - 2 a^2 n \eps t \le M_+ - \eps R^2 \le N_+ \label{kek}
  \end{equation}
  
  Последнее справедливо в силу того, что подходящее $R$ мы действительно можем подобрать. Таким образом, для всех достаточно маленьких $\eps$ в любом таком цилиндре подобное неравенство выполняется. \placeholder{Wut? Что произошло?}
  
  Устремляем $\eps \to 0$. Неравенство \eqref{kek} обратится в равенство. (\placeholder{почему?})
\end{proof}

\subsubsection{Начальная краевая задача}

Рассмотрим следующую задачу (здесь $t \in \mathbb{R}^+, x \in \Omega \subset \mathbb{R}^n$):
\begin{equation}
  \left\{\begin{aligned}
    Lu &= u_t - a^2 \Delta u = f \\
    u \eval_{\fr \Omega} &= \psi \\
    u \eval_{t = 0} &= u_0
  \end{aligned}\right.
\end{equation}
причем $f = f(x), \phi = \phi(x)$ --- не зависят от времени.

Но перед этим рассмотрим частный случай, когда $f \equiv 0$ и $\psi \equiv 0$.

\begin{thm}
  $u(t, x) \to 0$ (равномерно по $x$; с экспоненциальной скоростью).
\end{thm}

\begin{proof}
  НУО положим, что $\Omega$ включает $0$ и вложена в куб $Q$. Рассмотрим следующую функцию:
  \begin{equation}
    v(t, x) = Ae^{-bt} \prod_{i=1}^n \cos cx_i
  \end{equation}
  
  Применим к ней оператор теплопроводности:
  \begin{equation}
    Lv = \pdv{v}{t} - a^2 \Delta v = -bv + a^2 c^2 nv = v(-b + a^2 c^2 n) = 0
  \end{equation}
  
  Положим $\boxed{b \coloneqq a^2 c^2 n}$. Вощьмем такое $c$, что $\cos(cx_i) > 0$ для любых $x_i \in [-d, d]$. Дополнительно возьмем такое $A$, что:
  \begin{equation}
    A \prod_{i=1}^n \cos(cx_i) \ge \norm{u_0}_\infty
  \end{equation}
  
  Введем следующие обозначения:
  \begin{align}
    W_+(t, x) &\coloneqq v(t, x) + u(t, x) \\
    W_-(t, x) &\coloneqq v(t, x) - u(t, x)
  \end{align}
  
  Применим к ним оператор теплопроводности. Очевидно, $LW_\pm = 0$. Посмотрим на поведение $W_\pm$ при $t = 0$. Имеем:
  \begin{equation}
    W_\pm \eval_{t = 0} = v \eval_{t = 0} \pm u \eval_{t = 0} = A \prod_{i=1}^n \cos(cx_i) \pm u_0(x) \ge 0
  \end{equation}
  
  \placeholder{А тут я отвлекся. Какая печаль.}
\end{proof}

Теперь можем приступить к рассмотрению общего случая. \placeholder{Рассмотрим вспомогательное уравнение... Еще тут была формлировка теоремы (не одной?).}

\begin{proof}
  Пусть существуют два решения $u$ и $v$. Положим $w \coloneqq u - v$. Применим оператор:
  \begin{equation}
    Lw = Lu - Lv = 0
  \end{equation}
  
  Посмотрим на поведение на границе:
  \begin{equation}
    w \eval_{\fr \Omega} = u \eval_{\fr \Omega} - v \eval_{\fr \Omega} = 0
  \end{equation}
  и при $t = 0$:
  \begin{equation}
    w \eval_{t = 0} = u_0 - v
  \end{equation}
  
  Следовательно, $\abs{w(t, x)} \le Ae^{-bt}$, то есть $\abs{u(t, x) - v(x)} \le Ae^{-bt}$.
\end{proof}