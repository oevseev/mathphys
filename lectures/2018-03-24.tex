\dateline{24 марта 2018 г.}

\subsection{Обобщенные функции}

\placeholder{Опять пропущена первая пара. Да что ж такое...}

\bigskip

\placeholder{Напоминание: как интегрировать по частям в многомерном случае. Первая формула Грина.}

\paragraph{Второй пример}

Поменяем порядок:

\begin{equation}
  \begin{aligned}
    \int_\Omega v \cdot \Delta u \dd x = -\int_\Omega \Delta u \cdot \Delta v \dd x + \int_\Omega {\partial \Omega} v(x) \pdv{u}{n} (x) \dd \sigma(x)
  \end{aligned}
\end{equation}

Вычитая первое из второго, получаем

\begin{equation}
  \text{\placeholder{Еще одна выкладка}}
\end{equation}

\begin{cor*}
  Пусть $-\Delta u = f$ в $\Omega$, где $\Omega$ --- ограниченное множество с гладкой границей, $f$ --- непрерывная на $\bar{\Omega}$ функция, $u$ --- классическое решение ($u \in C^2(\bar{\Omega})$). Тогда:
  \begin{equation}
    \int_\Omega f \dd x = -\int_{\partial \Omega} \pdv{u}{n} \dd \sigma
  \end{equation}
  (<<количество тепла, которое выделилось, равно количеству тепла, ушедшего через границу>>)
\end{cor*}

\begin{cor*}
  \begin{equation}
    \left\{\begin{aligned}
      -\Delta u &= f\ \text{в $\Omega$, где $\Omega$ --- огр.} \\
      \pdv{u}{n} &= 0\ \text{на $\Omega$}
    \end{aligned}\right.
  \end{equation} --- задача Неймана.
  Классическое решение существует, если $\int_\Omega f \dd x = 0$.
\end{cor*}

\begin{thm*}
  Пусть $\Omega$ совпадает с $\mathbb{R}^n$, где $n \ge 3$. Тогда решением $-\Delta u = f$ в $D'(\Omega)$ является
  \begin{equation}
    u(x) = \frac{C_n}{\abs{x}^{n-2}}
  \end{equation}
\end{thm*}

\begin{proof}
  Рассмотрим $\phi \in D(\mathbb{R}^n)$:
  \begin{equation}
    \begin{aligned}
      \langle \phi, -\Delta u \rangle &= - \sum_{i=1}^n \left\langle \pdv{\phi}{x_i}{x_i}, u \right\rangle = - \langle \Delta \phi, u \rangle = \\
      &= -\int_{\mathbb{R}^n} \Delta \phi(x) u(x) \dd x = - \lim_{\eps \to 0} \int_{B_\eps^C(0)} \Delta \phi(x) u(x) \dd x \\
      &= -\lim_{\eps \to 0} \left(\int_{B_\eps^C(0)} \Delta u(x) \phi(x) \dd x + \int_{\partial B_\eps(0)} u(x) \pdv{\phi}{n} (x) \dd \sigma(x) \right)
    \end{aligned}
  \end{equation}
  
  \placeholder{Упражнение: проверить, что вне нуля классический лапласиан равен нулю.}
  
  \placeholder{Посмотреть выкладки у ребят...}
\end{proof}

Данная функция называется фундаментальным решением уравнения Лапласа. Классический же лапласиан всюду равен нулю.

\paragraph{Частный случай}

$n = 3$: $\omega_3 = \frac{4}{3} \pi$, $u(x) = \frac{1}{3 \cdot \frac{4}{3} \pi \cdot \abs{x}} = \frac{1}{4 \pi \abs{x}}$.

\begin{quote}
  Вообще иногда физики часто вместо $\delta$ пишут $\delta(x)$ <<как если бы это была функция>>, а вместо угловых скобок <<нахальным образом>> пишут интегралы, что с точки зрения математики не совсем корректно.
\end{quote}

\placeholder{Какое-то лирическое отступление про свертку}

Докажем один простой случай (теорема далеко не самая общая)
\begin{thm}
  Пусть $f \in C^2_0(\mathbb{R}^n)$. Тогда
  \begin{equation}
    u(x) = \int_{\mathbb{R}^n} \Phi(x-y) f(y) \dd y
  \end{equation}
  является \emph{классическим решением} уравнения Пуассона $-\Delta u = f$ в $\mathbb{R}^n$.
\end{thm}

\begin{proof}
  Во-первых, свертку можно перекинуть:
  \begin{equation}
    u(x) = \int_{\mathbb{R}^n} \Phi(y) f(x - y) \dd y
  \end{equation}
  
  Давайте считать соответственно $-\Delta u$:
  \begin{equation}
    -\Delta u = \int_{\mathbb{R}^n} \Phi(y) (- \Delta_x) f(x - y) \dd y
  \end{equation}
  Хотим доказать, что это равно $f$. Будем это делать в обобщенных производных (поскольку классические производные есть и непрерывны, они совпадут с обобщенными):
  \begin{equation}
    \begin{aligned}
      \langle \phi, - \Delta u \rangle &= -\int_{\mathbb{R}^n} \Phi(y) \dd y \int_{\mathbb{R}^n} \phi(x) \Delta_x f(x-y) \dd x \\
      &= -\int_{\mathbb{R}^n} \Phi(y) \dd y \int_{\mathbb{R}^n} \Delta \phi(x) f(x-y) \dd x \\
      &= -\int_{\mathbb{R}^n} \Phi(y) \dd y \int_{\mathbb{R}^n} \Delta \phi(y + z) f (z) \dd z \\
      &= -\int_{\mathbb{R}^n} -f(z) \dd z \int_{\mathbb{R}^n} (-\Delta \phi(y + z)) \Phi(y) \dd y && \text{--- теорема Фубини} \\
      &= -\int_{\mathbb{R}^n} -f(z) \dd z \langle \phi(\bullet + z), \delta \rangle && \text{\placeholder{а вот тут я не уверен}} \\
      &= \int_{\mathbb{R}^n} f(z) \phi(z) \dd z
    \end{aligned}
  \end{equation}
  
  Таким образом, $\langle \phi, -\Delta u\rangle = \langle \phi, f \rangle$, откуда $-\Delta u = f$.
\end{proof}