\dateline{28 апреля 2018 г.}

\placeholder{Пропущены первые 20 минут лекции.}

Если мы возьмем самосопряженный компактный оператор и рассмотрим любой конечный интервал $[-c, c]$, то вне него будет лежать не более чем конечное количество собственных чисел. Иными словами, у множества собственных чисел самосопряженного компактного оператора единственная возможная точка сгущения --- $0$.

Для оператора $\Delta$ у нас возможны две альтернативы: либо количество собственных чисел конечно, либо есть сколь угодно большое собственное число (потом мы узнаем, что первая альтернатива не реализуется).

Теперь поговорим о собственных функциях. Утверждение: если взять две любые собственные функции $u$ и $u'$, соответствующие двум \emph{разным} собственным числам $\mu$ и $\mu'$, то эти функции будут ортогональны друг другу.

\begin{thm*}(Гильберта-Шмидта)
  Пусть $A$ --- линейный ограниченный компактный самосопряженный оператор в $H$ --- сепарабельном гильбертовом пространстве. \placeholder{Ну тут понятно}
\end{thm*}

Теперь вернемся к нашему вопросу: какая из альтернатив для $\Delta$ имеет место быть? Принимая во внимание теорему Гильберта-Шмидта, становится очевидно, что вторая.

\placeholder{$\lambda_1$ всегда соответствует одномерное пространство собственных функций}

\placeholder{Что происходит?}

Раз $\mu u = Ku + g$, то $u$ можно представить как
\begin{equation}
  u = \sum_{i = 1}^\infty c_i u_i
\end{equation}
Кроме того, так как $g$ --- тоже элемент пространства, можем разложить и его:
\begin{equation}
  g = \sum_{i=1}^\infty d_i u_i
\end{equation}

Для простоты можем отнормировать \placeholder{что именно}. Давайте подставим это в уравнение \placeholder{(1')}:
\begin{equation}
  \sum_{i=1}^\infty \mu c_i u_i = \sum_{i=1}^\infty c_i Ku_i + \sum_{i=1}^\infty d_i u_i
\end{equation}

Можем объеденить это все под одним знаком суммы и получить следующее красивое соотношение:
\begin{equation}
  \sum_{i=1}^\infty c_i (\mu - \mu_i) u_i = \sum_{i=1}^\infty d_i u_i
\end{equation}

Справа и слева находятся какие-то функциональные ряды --- разложение одного и того элемента в ряд Фурье по базису $u_i$. Разложение по такому базису единственно, следовательно равенство может достигаться только при равенстве соответствующих коэффициентов. Так, для каждого $i$:
\begin{equation}
  c_i(\mu - \mu_i) = d_i
\end{equation}
откуда можем получить следующую формулу для коэффициентов разложения в ряд Фурье:
\begin{equation}
  \boxed{c_i = \frac{d_i}{\mu - \mu_i}}
\end{equation}

Разложение $u$ в ряд Фурье будет иметь следующий вид:
\begin{equation}
  u = \sum_{i=1}^\infty \frac{d_i}{\mu - \mu_i} u_i = \sum_{i=1}^\infty \frac{(g, u_i)}{\mu - \mu_i} u_i
\end{equation}

Так, если мы находимся в <<хорошем>> (нерезонансном) случае альтернативы Фредгольма, мы не только знаем, что наше решение уравнения \placeholder{(1)} устойчиво --- мы можем выписать для него явную формулу:
\begin{equation}
  u = \sum_{i=1}^\infty \frac{(-\Delta^{-1} f, u_i)}{-\lambda / \lambda_i} u_i
\end{equation}

Но что делать в нерезонансном случае, то есть, если $\lambda = \lambda_j$ (и соответственно, $\mu = \mu_j$)? Тогда \emph{все} $d_j$, соответствующие $\mu_j$, должны быть равны нулю. Иными словами, $(g, v) = 0$ для любой собственной функции $v$, соответствующей $\mu_j$.

\placeholder{Тут был какой-то итог}

Хочется научиться выбирать базис для пространства $H_0^1(\Omega)$. Как это делать?

Пусть $\qty{u_k}$ --- базис в $C(\Omega)$, но полная ОС он только в $L^2$ \placeholder{щито?}. Определим новое скалярное произведение в $H_0^1(\Omega)$:
\begin{equation}
  [u, w] \coloneqq \int_\Omega \nabla u \cdot \nabla w
\end{equation}

Мы уже вводили похожее понятие, когда определяли норму в $H_0^1(\Omega)$.

Распишем:
\begin{equation}
  [u_k, u_l] = \int_\Omega \nabla u_k \cdot \nabla u_l = -\int_\Omega u_k \Delta u_l
\end{equation}
Последнее равенство справедливо в силу \placeholder{чего-то, что было не разобрать на свежепомытой доске}

\begin{quotation}
  Извините, слишком много воды
\end{quotation}

\placeholder{НИХУЯ НЕ ВИДНО}

Давайте разделим $u_k$ на $\sqrt{\lambda_k}$ и назовем это $w_k$. Понятно, что $w_k$ --- это тоже базис, ортонормированный в смысле скалярного произведения $[\bullet, \bullet]$ и, следовательно, тоже базис в $L^2$.

\begin{thm*}
  $\qty{w_k}$ --- базис в $H_0^1(\Omega)$.
\end{thm*}

\begin{proof}
  \placeholder{Почему-то} осталось доказать, что $\qty{u_k}$ \placeholder{(sic!)} --- полная система, то есть если $\qty[u, w_k] = 0$ для любого $k$, то $u = 0$.
  
  Посчитаем $[u, w_k]$:
  \begin{equation}
    [u, w_k] = \int_\Omega \nabla u \cdot \nabla w_k = -\int_\Omega u \cdot \Delta w_k = -\sqrt{\lambda_k} \int_\Omega u u_k = 0
  \end{equation}
\end{proof}

\placeholder{Что-то пропущено?}

Если $h \in H_0^1(\Omega)$, то сходимость имеетя не только в $L^2(\Omega)$, но и в $H^1$ (то есть базис мы построили более сильный).

Поговорим о свойствах первого собственного числа.

\placeholder{кейс про мембрану барабана}

Взглянем на $u \in H_0^1(\Omega)$:
\begin{equation}
\begin{aligned}
  [u, u] &= \int_\Omega \abs{\nabla u}^2 = \int_\Omega \abs{\nabla \sum_k c_k u_k}^2 = \int_\Omega \abs{\sum_k c_k \nabla u_k}^2 \\
  &= \sum_k c_k^2 \int_\Omega \abs{\nabla u_k}^2 = \sum_k c_k^2 \lambda_k \ge \lambda_1 \sum_k c_k^2 = \lambda_1 \norm{u}_2^2
\end{aligned}
\end{equation}

\placeholder{небольшой пропуск}

\begin{thm*}
  \begin{equation}
    \lambda_1 = \min\{\int_\Omega \abs{\nabla u}^2, u \in H_0^1(\Omega), \norm{u}_2 = 1 \}
  \end{equation}
\end{thm*}

\placeholder{я заебался :(}