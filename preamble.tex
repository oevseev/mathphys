\usepackage[headsepline]{scrlayer-scrpage}

\usepackage{letltxmacro}
\usepackage{mathtools}
\usepackage{newclude}
\usepackage{physics}
\usepackage{xcolor}

% Структурные дефайны
\newtheorem{thm}{Теорема}
\newtheorem*{thm*}{Теорема}
\newtheorem{prop}[thm]{Утверждение} 
\newtheorem*{prop*}{Утверждение}
\newtheorem{cor}{Следствие}
\newtheorem*{cor*}{Следствие}

\theoremstyle{definition}
\newtheorem*{defn}{Определение}

\theoremstyle{remark}
\newtheorem*{rem}{Замечание}
\newtheorem*{rems}{Замечания}
\newtheorem*{example}{Пример}
\newtheorem*{examples}{Примеры}
\newtheorem*{idea}{Идея доказательства}

\newtheoremstyle{lemma}{}{}{}{}{\bfseries}{.}{.5em}{}
\theoremstyle{lemma}
\newtheorem{lem}{Лемма}
\newtheorem*{lem*}{Лемма}

\newtheoremstyle{problem}{}{}{}{}{\bfseries}{.}{.5em}{}
\newtheorem{problem}{Задача}
\newtheorem{problem*}{Задача}

% Сброс нумерации теорем и следствий
\makeatletter
\@addtoreset{thm}{subsection}
\@addtoreset{cor}{thm}
\makeatother

% Нумеруем леммы кириллицей
\renewcommand{\thelem}{\Asbuk{lem}}

% Русские эпсилон и фи
\LetLtxMacro{\temp}{\epsilon}
\LetLtxMacro{\epsilon}{\varepsilon}
\LetLtxMacro{\varepsilon}{\temp}
\LetLtxMacro{\temp}{\phi}
\LetLtxMacro{\phi}{\varphi}
\LetLtxMacro{\varphi}{\temp}

% shorthand for epsilon
\newcommand{\eps}{\epsilon}

% Полезные определения
\newcommand{\nil}{\mathbf{0}}
\newcommand{\placeholder}[1]{{\color{gray}#1}}

% Стиль страницы
\clearpairofpagestyles
\pagestyle{scrheadings}
\cfoot[\pagemark]{\pagemark}

% Красивые пустые страницы (с заглушкой) после /cleardoublepage, если нужно
% (украдено с https://tex.stackexchange.com/q/1727)
\newcommand*{\blankpage}{%
  \vspace*{\fill}
  {\centering\itshape Эта страница намеренно оставлена пустой.\par}
  \vspace{\fill}}
\makeatletter
\renewcommand*{\cleardoublepage}{%
  \clearpage
  \if@twoside
    \ifodd
      \c@page
    \else
      \blankpage
      \thispagestyle{plain}
      \newpage
      \if@twocolumn
        \hbox{}
        \newpage
      \fi
    \fi
  \fi}
\makeatother

% Линия для разделения лекций
% (украдено с https://tex.stackexchange.com/questions/15119)
\def\dateline#1{%
  {\color{gray}
  \hbox to \hsize{%
    \hrulefill\quad\lower.3em\hbox{#1}\quad\hrulefill
  }}
  \vskip .7em
}